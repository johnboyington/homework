\documentclass[11pt]{article}
\usepackage{mathtools,hyperref,booktabs,fullpage}
%\usepackage[amssymb,cdot]{SIunits}
%\usepackage[utopia]{mathdesign}     
\usepackage{xcolor}
\usepackage{amsmath}
\usepackage{amssymb}
\usepackage{hyperref}
\usepackage{longtable}
\usepackage{fullpage}
\usepackage{enumitem}
\setlist{nolistsep}

%\definecolor{lightgray}{gray}{0.93}

\pagestyle{empty}
\setlength\parindent{0pt}
\renewcommand{\thefootnote}{\fnsymbol{footnote}}
 
\makeatletter
\renewcommand\section{\@startsection{section}{1}{\z@}%
                                  {-3.5ex \@plus -1ex \@minus -.2ex}%
                                  {2.3ex \@plus.2ex}%
                                  {\normalfont\bfseries}}
\makeatother


\begin{document}

{\large
  \begin{center}
    {\bf ME 701 -- Development of Computer Applications In Mechanical Engineering \\ 
         Homework 9 --Due 11/27/2017 \\
    }
  \end{center}

\setlength{\unitlength}{1in}

\begin{picture}(6,.1) 
\put(0,0) {\line(1,0){7.35}}         
\end{picture}
}

{\bf Instructions}:  All source code should be placed in a single 
TAR file of the form {\tt lastname\_firstname\_hw8.tar}.  All 
code should be compiled with a Makefile, as specified in Problem 3.

\section*{Problem 1 -- Just One!}

Consider the equation
\begin{equation}
  \frac{d}{dx} a(x) \frac{d y}{dx} + b(x) y(x) = c(x) 
\end{equation}
subject to $y(L) = y_L$ and $y(R) = y_R$.  Here, $a$, $b$, and $c$ are
arbitrary functions of $x$ provided in {\it tabulated} format. For simplicity,
assume that $a$, $b$, and $c$ are defined for the same values of $x$.\\

Write a
function or subroutine in C++ or Fortran with the following signature
\begin{verbatim}
void solve(int n, double L, double R, double y_L, double y_R, 
           int m, double *a, double *b, double *c,
           int option=0, double tol = 1e-8, int maxit = 100);
// similar for fortran
\end{verbatim}
where {\tt n} is the number of equally spaced $x$ points at which $y(x)$ is 
to be evaluated, $L$ and $R$ are
the left and right boundaries, $y_L$ and $y_R$ are the left and right
boundary conditions, and {\tt a}, {\tt b}, and {\tt c} are double arrays
evaluated at {\tt m} equally spaced points from {\tt L} to {\tt R}.
The argument {\tt option} is an integer for which {\tt 0} means
use tridiagonal elimination, {\tt 1} means use Jacobi,  {\tt 2} means use
Gauss-Seidel, and {\tt 3} means use any other iterative method of your 
choice (+1/2 point).  Finally, {\tt tol} and {\tt maxit} define the tolerance
and maximum number of iterations for options 1, 2, and 3.  \\

You may wish to produce a {\tt main.cc} for testing, or to couple this
with Python via {\tt f2py} or {\tt swig}.  All I want is your function
or subroutine code (include a header if doing C++).\\

{\bf Note!!!} I strongly encourage you to work out the finite difference 
(or any other approximation you want to use) before lecture on 11/17 so that
I can tell you if you're on the right track.\\

{\bf Hint!!!} Have a reference solution handy for debugging and method
verification!  (Solve, e.g., $y'' = 1$ with $y(0) = y(1) = 0$, etc.).

\end{document}
